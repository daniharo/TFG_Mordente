\chapter{Introducción}

Este proyecto intenta dar respuesta a las agrupaciones musicales que demandan una plataforma gratuita y libre para la mejora de su gestión.

\section{Motivación y justificación del proyecto}

% ¿Qué necesidad hay?
% ¿Qué problema existe que se quiera resolver?
% ¿Qué mejoras propongo sobre algo ya existente?
% ¿Por qué he elegido esta temática?

En 2018 se contabilizaron unas 6197 agrupaciones musicales en España según el INE \cite{ineNumeroAgrupaciones}, número que se mantiene estable a lo largo de los años. Cada agrupación musical, definida como ``dos o más personas que, a través de la voz o de instrumentos musicales, interpretan obras musicales pertenecientes a diferentes géneros y estilos'' según la RAE, se compone de distintos miembros, formando una sociedad. Como en todas las sociedades, se necesita cierto orden para poder alcanzar los objetivos, por lo que algunos de los miembros se tienen que encargar de su gestión: el director musical, el director artístico, el presidente de la asociación, el secretario... Siendo la gestión distinta en cada agrupación.

Los miembros de las agrupaciones musicales realizan ensayos periódicamente con el doble objetivo de mejorar su técnica de interpretación y preparar la celebración de eventos, tales como conciertos, certámenes, concursos, pasacalles... En los ensayos y eventos se interpreta una o varias obras musicales.

El director musical de una agrupación musical se encarga de coordinar la interpretación de las obras musicales, así como planificar y preparar los ensayos y eventos. En ocasiones se sustituye o se complementa con el rol de director artístico, visibilizando el hecho de que los eventos a menudo no son puramente musicales, sino que se añaden teatros o espectáculos visuales como bailes o musicales.

Los administradores de la agrupación (a menudo llamados junta directiva) se encargan de las labores de administración no musicales, ayudando también al director. Habitualmente entre estos encontramos un presidente, un secretario y un tesorero.

Para interpretar una obra musical, los miembros de la agrupación leen un papel. El papel puede ser de dos tipos: el que lee el director se denomina \textit{partitura}, y es el papel que recoge la música de todos los instrumentos y voces. Por otra parte, cada músico lee su \textit{particella} (también llamada ``parte''), que solo recoge lo que debe interpretar él individualmente.

En los últimos tiempos la gestión de estas agrupaciones se ha realizado de forma manual, consistente en que los administradores, el director y los miembros se comunican mediante un grupo de mensajería sin ningún tipo de automatización.

De esta forma, algunas de las tareas más frecuentes en la administración de la agrupación podrían ser:

\begin{itemize}
    \item Para la asistencia a eventos y ensayos, cada miembro escribe a través de un grupo, o individualmente al director o un administrador, su intención de asistir o no. Esta comunicación previa le es útil para gestionar qué repertorio es mejor ensayar y en qué orden, o si por el contrario es conveniente aplazar o suspender el evento o ensayo.
    \item Para el repertorio, los distintos documentos (partitura y partes) que necesitan los músicos se envían a través de alguna plataforma de almacenamiento en la nube (como Google Drive, OneDrive o Dropbox) o se imprimen en papel físico por los administradores, repartiéndose en un ensayo a los miembros presentes.
    \item Respecto a la comunicación de eventos y ensayos, se realiza por el canal no especializado de comunicación y la responsabilidad de recordarlos debidamente a los miembros antes de su celebración recae en los administradores.
\end{itemize}

Esta gestión manual y dependiente de aplicaciones de comunicación no especializadas deriva en varios problemas:

\begin{itemize}
    \item La probabilidad de error humano aumenta en el momento en el que los administradores son responsables de recordar eventos y gestionar asistencia.
    \item La carga de trabajo de los administradores es mayor dada la responsabilidad de gestión manual que tienen a la hora de organizar un evento.
    \item Se crea una duplicidad de comunicación, ya que los miembros pueden escribir sobre su asistencia al director o a los administradores, pudiendo darse el caso de que finalmente no todos dispongan de los datos necesarios para organizar un ensayo o evento, o que dispongan de datos contradictorios entre sí.
\end{itemize}

Por todos estos problemas derivados de la gestión manual se necesita software dedicado específicamente a este propósito, automatizando la gestión, que sea personalizable a las necesidades de cada agrupación, no implique enormes gastos de uso y sea fácilmente usable por todos los usuarios.

De este modo, el proyecto está justificado por la necesidad de alternativas para mejorar la gestión del gran número de agrupaciones musicales que existen en nuestro país y que no han podido automatizar aún su gestión.

Por último, dada la conveniencia de realizar pruebas con usuarios reales, el proyecto va a contar con la colaboración de la Banda de la Agrupación Músico-Cultural ``San Sebastián'' de Padul, con unos 80 miembros.

\section{Objetivos}

% - 1 objetivo general pero que concrete lo que se pretende hacer
% - Varios objetivos específicos redactados en infinitivo y con justificación.

Los objetivos de este trabajo se pueden resumir en tres fundamentales:

\begin{itemize}
    \item Por un lado, dar respuesta a los requerimientos de numerosas agrupaciones musicales con un software que les permita automatizar la gestión de eventos, miembros, ensayos, repertorio, etc., ayudando a una organización más ágil y efectiva.
    \item Por otro lado, investigar hasta qué punto la interfaz de usuario proporcionada por un bot de Telegram es suficiente, competitiva y cómoda para el usuario. Esta interfaz está integrada en un chat, por lo que se pretende identificar las limitaciones que puedan darse, y explorar soluciones que sorteen dichas carencias.
    \item Aportar al software libre no solo el código del bot desarrollado sino también bibliotecas que se integren en otros bots y que puedan ser usadas directamente por la comunidad.
\end{itemize}

Para el primer objetivo se cumpla de manera satisfactoria, se establecen varios requisitos básicos:

\begin{itemize}
    \item Se debe disponer de una página web donde quede claro qué problemas pretende resolver la herramienta, de modo que las agrupaciones puedan ver las ventajas de su uso. Esta información también debe quedar clara en la propia herramienta.
    \item El uso de la herramienta debe ser lo más sencillo y obvio posible, dada la heterogeneidad de los usuarios, por su edad, profesión, nivel de estudios o grado de implicación en la agrupación.
    \item En el desarrollo de la herramienta se tienen que tener en cuenta los roles de miembro no administrador y quien sí es administrador, de forma que cada rol deberá tener distintas funcionalidades disponibles. Por ejemplo, solo un administrador podrá eliminar a otros miembros.
    \item Se debe buscar el mejor diseño que se pueda ofrecer dentro de las limitaciones de la tecnología escogida, de modo que un mal diseño no haga que los usuarios prefieran volver a la gestión manual.
\end{itemize}

%%Rosana: muy buenos requisitos básicos, o "fundamentos del proyecto"