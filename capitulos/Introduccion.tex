\chapter{Introducción}

\section{Motivación y justificación del proyecto}

% ¿Qué necesidad hay?
% ¿Qué problema existe que se quiera resolver?
% ¿Qué mejoras propongo sobre algo ya existente?
% ¿Por qué he elegido esta temática?

A la hora de gestionar una agrupación musical, ya sea por parte de su director o de los administradores, se recurre mayoritariamente a herramientas de comunicación no especializadas, principalmente \textit{WhatsApp}. Así, la gestión de una agrupación consiste en:

\begin{itemize}
    \item Para la asistencia, cada miembro escribe a través de un grupo, o al director, si no puede asistir a un ensayo o evento.
    \item Para el repertorio, los distintos documentos (partitura y partes) que necesitan los músicos se envían a través de alguna plataforma de cloud (Google Drive, Onedrive, Dropbox) o se imprimen exclusivamente por los administradores.
    \item Respecto a los eventos y ensayos, se avisan por el canal no especializado de comunicación y la responsabilidad de recordarlos debidamente antes de su celebración recae en los administradores.
\end{itemize}


Esto deriva en varios problemas:

\begin{itemize}
    \item La probabilidad de error humano aumenta en el momento en el que los administradores son responsables de recordar eventos y gestionar asistencia.
    \item Se crea una duplicidad de comunicación, ya que los miembros pueden escribir sobre su asistencia al director o a los administradores, pudiendo darse el caso de que finalmente no todos dispongan de los datos necesarios para organizar un ensayo u evento, o que dispongan de datos contradictorios entre sí.
\end{itemize}

En los últimos años ha surgido software dedicado a este propósito, como es el caso de \textit{Glissandoo}. Sin embargo, esta herramienta tiene varias desventajas:

\begin{itemize}
    \item El código es privado, y una agrupación no puede montar su propio servidor con la aplicación en cuestión.
    \item Es de pago a partir de 20 miembros.
    \item Solo se puede usar en móvil, ya que no dispone de versión web ni de escritorio.
    \item Sigue existiendo duplicidad, ya que aunque los miembros usen esta aplicación, la comunicación bidireccional entre administradores y miembros de la banda se sigue realizando a través de una herramienta de comunicación no especializada.
\end{itemize}

En este trabajo se propone una alternativa libre que pueda resolver los problemas anteriores, de modo que:

\begin{itemize}
    \item Cada agrupación podrá montar su propio servidor con la aplicación, personálizandola y adaptándola a sus necesidades.
    \item Los gastos por uso de la herramienta solo dependerán del servidor que use la agrupación, en su caso.
    \item Será usable en todas las plataformas en las que se pueda acceder a un navegador web.
    \item Estará integrada en una herramienta de comunicación no especializada como es Telegram, pero añadiendo funcionalidad específica y automática para la gestión de la agrupación.
\end{itemize}



\section{Objetivos}

% - 1 objetivo general pero que concrete lo que se pretende hacer
% - Varios objetivos específicos redactados en infinitivo y con justificación.

Los objetivos de este trabajo se pueden resumir en dos fundamentales:

\begin{itemize}
    \item Por un lado, dar respuesta a los requerimientos de numerosas agrupaciones musicales de diversos tipos, como orquestas, bandas, charangas, etc, de un software que les permita gestionar eficientemente eventos, miembros, ensayos, repertorio, etc.
    \item Por otro lado, investigar hasta qué punto la interfaz de usuario proporcionada por un bot de Telegram es suficiente, competitiva y cómoda para el usuario. Esta interfaz está integrada en un chat, por lo que se pretende identificar las limitaciones que puedan darse, y explorar soluciones para evitar dichas carencias.
\end{itemize}


