\chapter{Pruebas}

% hacer ya DISEÑO PRUEBAS y diseño interfaz

% primera entre el 2 y 4

% segunda 10




% pruebas de usabilidad, (modificaciones) y de software

% net promoter score
% describir participantes, métricas
% describir resultado de cada tarea

Tras el capítulo anterior, ya tenemos una implementación de nuestra herramienta que los usuarios pueden probar para darnos retroalimentación y opiniones.

Las pruebas que se van a realizar en este trabajo, dado que la metodología usada ha sido el Design Thinking y el Diseño Centrado en el Usuario, serán pruebas de usabilidad con usuarios reales.

\section{Diseño de las pruebas}

En esta sección se va a determinar cuál será el diseño de las pruebas de usabilidad.

\subsection{Participantes}

Dado que hay dos roles claramente distinguibles en la aplicación, se seleccionarán 3 administradores y 7 miembros para completar la prueba de usabilidad.

\subsection{Tareas a realizar}

En el caso de los administradores:

\begin{enumerate}
    \item Crea una agrupación.
    \item Crea un evento.
    \item Sube una obra a la agrupación.
    \item Invita a un miembro de tu banda.
    \item Haz que no pueda unirse nadie más a la agrupación.
    \item Haz administrador al miembro que se ha unido.
    \item Quítale los permisos de administrador al miembro que se ha unido.
    \item Elimina el evento que creaste.
\end{enumerate}

En el caso de los miembros:

\begin{enumerate}
    \item Únete a una agrupación.
    \item Mira los eventos de la agrupación.
    \item Descarga alguna obra de la agrupación.
    \item Sal de la agrupación.
\end{enumerate}


\subsection{Preguntas a realizar}

Para cada tarea, se propondrán las siguientes preguntas:
\begin{enumerate}
    \item ¿Has podido realizar la tarea?
    \item ¿Ha sido fácil realizar la tarea?
\end{enumerate}

Finalmente, se añadirán dos preguntas genéricas:

\begin{enumerate}
    \item ¿Recomendarías la aplicación a un amigo? Entre 0 y 10.
    \item ¿Tienes algún comentario? Respuesta libre.
\end{enumerate}

\section{Realización de las pruebas}


\section{Informe final de las pruebas}


\section{Demostración en vídeo}

Se puede visualizar el funcionamiento del bot en el vídeo disponible en este enlace:
\url{}

