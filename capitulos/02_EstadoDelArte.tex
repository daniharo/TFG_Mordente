\chapter{Estado del arte}

% Descripción de los antecedentes del trabajo.
% Elegir un título apropiado.

\section{Dominio del problema a resolver}

En la actualidad, la digitalización de empresas, administraciones y organismos públicos avanza a ritmo imparable, y la situación excepcional generada por la pandemia de COVID-19 no ha hecho más que afianzar este avance.

Los datos del Índice de Digitalización de la Economía y la Sociedad (DESI) en 2022\cite{desiUE}, índice que monitoriza el rendimiento digital de Europa y el progreso de los países de la Unión Europea en su competitividad digital, muestran que todos los países han progresado en su digitalización. Destaca el hecho de que los países que han empezado en un nivel más bajo de desarrollo digital crecen a un ritmo más rápido, y que numerosos países crecen a un ritmo más elevado del que se les espera, como es el caso de España\cite{desiSpain} (con un 0.7\% más de progreso del esperado).
En concreto, nuestro país es el séptimo con mayor nivel de digitalización según este índice, y el número de personas con competencias digitales básicas es del 64\%, por encima de la media de la UE, del 54\%. Está por encima de la UE también en implementación de la banda ancha fija de al menos 100 Mbps, usuarios de la administración electrónica o pymes con nivel básico de intensidad digital.

%España 2º lugar Índice de Digitalización de la Economía y la Sociedad

%Otras áreas se han modernizado, como todas las empresas. Datos: https://servicioestudiosugt.com/digitalizacion-de-la-empresa-espanola-3-edicion/
A nivel empresarial otros estudios concluyen que la digitalización está más estancada\cite{digitalizacionEmpresaUGT}, debido a una baja inversión en TIC.

%Incluso a nivel de la administración se están impulsando estrategias para avanzar en la digitalización, como España Digital 2026
Por otro lado, a nivel de la administración se están impulsando estrategias para avanzar en la digitalización, como España Digital 2026\cite{espanaDigital2026}, fomentando programas para acelerar la digitalización de pymes (pequeñas y medianas empresas), mejorar la conectividad o 

%Sin embargo, las agrupaciones musicales, que aumentan, siguen sin modernizarse, cuando podrían disfrutar de numerosas ventajas haciendo uso de la tecnología.

Pese a todo esto, algunos sectores de la sociedad aún manifiestan reticencias para proceder a su digitalización, por la falta de herramientas, recursos o ayuda. 
Una de ellas es la que nos ocupa en este trabajo: las asociaciones o agrupaciones musicales. La complejidad de su gestión es suficiente para que los beneficios de la digitalización se puedan manifestar, dada la cantidad de personas que se pueden encontrar en cada una de ellas y la naturaleza de los procesos que siguen, como la programación de ensayos y eventos, la gestión de repertorio o la gestión de miembros.

Por ejemplo, para la correcta planificación de ensayos y eventos, tanto el director como los administradores necesitan conocer con antelación qué músicos tienen pensado asistir y cuáles estarán ausentes. Esto les permitirá saber si se necesita llamar a un músico externo a la agrupación para que participe en un evento, o si es mejor cancelar o aplazar el evento. Para esta tarea, habitualmente el director avisa del evento por un grupo de mensajería a los miembros, y pide que quien no tenga pensado asistir se lo notifique. De este modo se generan varios problemas:

\begin{itemize}
    \item La información sobre qué miembros han notificado su ``no asistencia'' solo la tiene el director, por lo que si otros administradores quieren conocerla se la tienen que pedir individualmente.
    \item Si algún miembro ha comunicado su ``no asistencia'' a otro administrador, tiene que poner la información en común con la del director, pudiendo dar lugar a discrepancias.
    \item Si el director quiere tener la información de asistencia centralizada, tiene que repasar sus conversaciones con los miembros para elaborar una tabla de asistencia, que tiene que ir actualizando manualmente, con la carga de trabajo que ello supone.
\end{itemize}

Similares problemas se crean para los demás procesos que se llevan a cabo en una agrupación musical.

Es evidente, por tanto, que se necesita una herramienta específica que permita automatizar sus procesos y avanzar en su digitalización para equipararla a la del resto de la sociedad.

Aunque ya existen herramientas como Glissandoo, que se analiza en este capítulo, son de pago, poco personalizables o están poco integradas en el flujo de trabajo actual de las agrupaciones. Es por ello que este trabajo pretende aportar una alternativa libre que solucione estos inconvenientes, de modo que:

\begin{itemize}
    \item Cada agrupación pueda montar su propio servidor con la aplicación, personálizandola y adaptándola a sus necesidades.
    \item Los gastos por uso de la herramienta solo dependan del servidor que use la agrupación, si usa un servidor dedicado.
    \item Sea usable en todas las plataformas en las que se pueda acceder a un navegador web.
    \item Esté integrada en una herramienta de mensajería que ya esté en uso como es Telegram, pero añadiendo funcionalidad específica y automática necesaria para la gestión de la agrupación, eximiendo a los administradores y el director de tareas que se pueden automatizar.
\end{itemize}


% Descripción más detallada del problema, carencias que existen,
% características específicas de los usuarios finales, descripción de
% trabajos previos del cual este es una extensión o de proyectos de 
% investigación en el que se enmarca, etc.

\section{Trabajos relacionados}
% Evidencias científicas o revisiones recientes de uso de tecnologías o
% metodologías para resolver un problema igual o similar al mío. Buscar
% referencias en fuentes fiables como Google Scholar.

\section{Aplicaciones similares}
% Descripciones, capturas y tabla comparativa
% Alternativas: Glissandoo, Whatsapp

Aunque aún la mayoría de agrupaciones usan Whatsapp como herramienta de comunicación sin usar ningún software más que automatice las tareas, en los últimos tiempos ha surgido otra herramienta que da respuesta a sus necesidades:

\subsection{Glissandoo}

Glissando (\url{https://glissandoo.com/}) es un software creado por la empresa Plausible Technologies, con sede en Valencia. En su página web se define como ``un software que nace con el objetivo de profesionalizar, modernizar y digitalizar las instituciones musicales''. Incorpora numerosas funcionalidades para digitalizar instituciones musicales, tal y como pretende este trabajo. Entre ellas se encuentran:

\begin{itemize}
    \item Organización de ensayos y conciertos. Los miembros pueden ver los ensayos y conciertos en la página de inicio.
    \item Previsión de asistencia y pasar lista: para cada ensayo o concierto, el miembro puede seleccionar si tiene prevista su asistencia o no.
    \item Distribución de partituras: en la sección de inicio se puede acceder a un listado de todo el repertorio de los ensayos y conciertos programados, y en cada uno de los ensayos se puede ver el repertorio asociado.
    \item Comunicaciones: los administradores pueden añadir avisos que reciben los miembros, teniendo la capacidad de responder, reaccionar, y adjuntar archivos a los comunicados.
\end{itemize}

Se pueden ver varias capturas de la aplicación en la figura \ref{fig:capturas_glissandoo}.
\begin{figure}[h]
\minipage{0.32\textwidth}
  \includegraphics[width=\linewidth]{imagenes/capturas_glissandoo/IMG_0928.jpeg}
\endminipage\hfill
\minipage{0.32\textwidth}
  \includegraphics[width=\linewidth]{imagenes/capturas_glissandoo/IMG_0929.jpeg}
\endminipage\hfill
\minipage{0.32\textwidth}
  \includegraphics[width=\linewidth]{imagenes/capturas_glissandoo/IMG_0930.jpeg}
\endminipage
\caption{Capturas de la aplicación \textit{Glissandoo}}\label{fig:capturas_glissandoo}
\centering
\end{figure}

Algunas ventajas de esta alternativa son:

\begin{itemize}
    \item Presenta un diseño intuitivo y fácil de usar para todos los usuarios.
    \item Está disponible para las plataformas móviles iOS y Android, ampliamente usadas por los usuarios.
    \item Es gratuita si la agrupación tiene menos de 20 músicos.
    \item Es elaborada en por una empresa valenciana, lo cual da seguridad de que se adaptará a las necesidades de las agrupaciones musicales españolas.
    \item Incorpora un módulo de comunicaciones para poder centralizar todos los procesos de la agrupación en la aplicación, sin dejar los comunicados en otra aplicación de mensajería como WhatsApp.
\end{itemize}

Sin embargo, esta herramienta tiene varias desventajas:

\begin{itemize}
    \item El código es privado, y una agrupación no puede montar su propio servidor con la aplicación en cuestión.
    \item Es de pago para uso en agrupaciones con más de 20 miembros.
    \item Solo se puede usar en móvil, ya que no dispone de versión web ni de escritorio. (Esto ha cambiado durante la realización de este trabajo, ya que han implementado una versión web)
    \item Sigue existiendo duplicidad, ya que aunque los miembros usen esta aplicación, la comunicación bidireccional ocasional entre administradores y miembros de la banda se sigue realizando a través de una herramienta distinta de comunicación no especializada.
\end{itemize}


\section{Metodologías y tecnologías de base que podrían usarse}

\subsection{Lenguaje de programación}
Se plantean los siguientes lenguajes de alto nivel, que disponen de frameworks implementados para crear un bot de Telegram:

\begin{itemize}
    \item \textbf{JavaScript}: es un lenguaje compilado en tiempo de ejecución, con tipado dinámico y multi-paradigma, soportando programación orientada a objetos, funcional, imperativa y dirigida por eventos\cite{wiki:JavaScript}. Se conforma al estándar ``ECMAScript'', y es una de las tecnologías centrales de la World Wide Web: el 98\% de los sitios web lo usan en el lado del cliente\cite{javascriptUsage}, y desde el surgimiento de Node.js es una tecnología en auge para servidores.
    \item \textbf{TypeScript} (\url{https://www.typescriptlang.org/}): es un superconjunto de JavaScript que añade sintaxis para tipos estáticos, y a través de un compilador genera código JavaScript\cite{typescriptWeb}. El añadido de tipos estáticos permite detectar errores de forma más temprana y agilizar la escritura de código gracias al autocompletado. Por otro lado, la adición de tipos es un tiempo extra empleado por el programador, por lo que debe analizarse si es conveniente su uso o no.
    \item \textbf{Python} (\url{https://www.python.org/}): es un lenguaje interpretado, interactivo y principalmente orientado a objetos, aunque soporta otros paradigmas como el funcional o el procedimental\cite{pythonFAQGeneral}. Es muy usado en el campo de la computación científica y la inteligencia artificial. Su uso en el desarrollo web se ha extendido para el lado del servidor con la aparición de frameworks como Django (\url{https://www.djangoproject.com/}) o Flask (\url{https://flask.palletsprojects.com/}).
    \item \textbf{PHP} (\url{https://www.php.net/}): es un lenguaje interpretado usado principalmente para desarrollo web en el lado del servidor, cuya principal característica es que puede ser embebido en HTML, de forma que cada ``trozo de PHP'' se ejecuta para intercambiarse por HTML. Sin embargo su uso ha cambiado en los últimos años, dando lugar a frameworks como Symphony (\url{https://symfony.com/}) o Laravel (\url{https://laravel.com/}) basados en la arquitectura Modelo-vista-controlador.
\end{itemize}

Se proporciona una tabla comparativa entre los lenguajes, la tabla \ref{tab:comparacionLenguajes}.

\begin{table}
\begin{minipage}{\textwidth}
\begin{tabularx}{\textwidth}{|l|X|X|X|X|}
\hline
                                                    & JavaScript                       & TypeScript             & Python                    & PHP                       \\
\hline
Tipo                                                & Compilado en tiempo de ejecución & Compilado a JavaScript & Interpretado              & Interpretado              \\
\hline
Tipado estático                                     & No                               & Sí                     & Poco estricto, y poco uso & Poco estricto, y poco uso \\
\hline
Coste\footnote{Aprendizaje necesario por parte del autor para poder implementar el proyecto}                                & Nulo                             & Bajo                   & Medio                     & Medio                     \\
\hline
Popularidad\footnote{\url{https://survey.stackoverflow.co/2022/}} & 65.36\%                          & 34.83\%                & 48.07\%                   & 20.87\%                   \\
\hline
Comunidad \footnote{Preguntas totales en stackoverflow: \url{https://stackoverflow.com/tags}}      & 2432281                          & 196707                 & 2035248                   & 1447104                  &
\hline
\end{tabularx}
\end{minipage}
\caption{Comparación de lenguajes de programación}\label{tab:comparacionLenguajes}
\end{table}


\subsection{Framework para el desarrollo del bot}

Existen diversas bibliotecas que ayudan a crear un bot, de forma que no haya que escribir todo el código desde cero para interaccionar con la API de Telegram, siguiendo el principio ``Don't Reinvent the Wheel'' (``No reinventes la rueda'') para también ahorrar presupuesto.

Las opciones que se analizan son:

\begin{itemize}
    \item \textbf{Telegraf} (\url{https://telegraf.js.org/}): biblioteca desarrollada inicialmente para JavaScript, migrada en la versión 4 dando soporte a TypeScript. No dispone de documentación guiada, y la migración a la versión 4 trajo consigo más complejidad de uso. Está inspirada el sistema de middleware de Express.\footnote{\url{https://expressjs.com/es/guide/using-middleware.html}}
    \item \textbf{grammY} (\url{https://grammy.dev/}): biblioteca desarrollada desde un inicio para TypeScript (aunque se puede usar con JavaScript). Está inspirada en telegraf y fue creada desde cero por uno de los antiguos mantenedores de Telegraf como única forma de resolver sus mayores inconvenientes\footnote{\url{https://github.com/telegraf/telegraf/discussions/1526}}. Dispone de una completa documentación con guías para cada uno de los conceptos importantes.
    \item \textbf{node-telegram-bot-api} (\url{https://github.com/yagop/node-telegram-bot-api}): biblioteca ligera para interactuar con la API de Telegram, diseñada para Node.js, y no implementa un sistema de middleware.
    \item \textbf{python-telegram-bot} (\url{https://python-telegram-bot.org/}): biblioteca para Python que provee clases de alto nivel como capa de abstracción sobre la API de Telegram.
\end{itemize}

La tabla \ref{tab:comparacionFrameworks} compara los distintos frameworks.

\begin{table}
\begin{minipage}{\textwidth}
\begin{tabularx}{\textwidth}{|l|X|X|X|X|}
\hline
& telegraf & grammY & node-telegram-bot-api & python-telegram-bot \\
\hline
Lenguaje & JavaScript & Typescript & JavaScript & Python \\
\hline
Documentación & Autogenerada, solo API & Completa, numerosos tutoriales & Básica & Autogenerada, solo API \\
\hline
\makecell[l]{Versión API \\Telegram} & 6.2 & 6.2 & 6.2 & 6.2 \\
\hline
Popularidad\footnote{Estrellas en GitHub a 07/10/2022} & 6.1k & 663 & 6.5k & 19.8k \\
\hline
\end{tabularx}
\end{minipage}
\caption{Comparación de frameworks para creación de bots de Telegram}\label{tab:comparacionFrameworks}
\end{table}

\subsection{Base de datos}

La base de datos es ``una colección compartida de datos lógicamente relacionados, junto con una descripción de estos datos, que están diseñados para satisfacer las necesidades de información de una organización''.\cite{alma991009264529704990}

Por otra parte, el sistema gestor de base de datos (SGBD) es ``un sistema software que permite a los usuarios definir, mantener, crear y controlar el acceso a la base de datos.\cite{alma991009264529704990}

Necesitaremos estas herramientas ya que buscamos un sistema que guarde datos de manera persistente y poder recuperarlos y actualizarlos en cualquier momento.

Se analizan alternativas para bases de datos relacionales, basadas en el concepto matemático de relación, representado por tablas\cite{alma991009264529704990}; y también para bases de datos no relacionales documentales.

\begin{itemize}
    \item Para bases de datos relacionales:
    \begin{itemize}
        \item PostgreSQL (\url{https://www.postgresql.org/}): es un SGBD de código abierto y centrado en la escabilidad.
        \item MySQL (\url{https://www.mysql.com/}): es un SGDB de código abierto desarrollado por Oracle.
        \item MariaDB (\url{https://mariadb.org/}): es un SGDB de código abierto creado por desarrolladores de MySQL como respuesta a la adquisición por parte de Oracle, y con características muy parecidas.
    \end{itemize}
    \item Para bases de datos no relacionales:
    \begin{itemize}
        \item MongoDB (\url{https://www.mongodb.com/}): SGDB de código disponible (no necesariamente de código abierto, por usar una licencia SSPL.\cite{ssplLicense}
        \item Firestore (\url{https://firebase.google.com/docs/firestore}): incluido en la suite de servicios ``Firebase'' de Google destinada a la gestión rápida y centralizada de aplicaciones.
    \end{itemize}
\end{itemize}


% Puesto que las diferencias son pequeñas entre las opciones de base de datos relacional y la documentación de Prisma está mayormente basada en PostgreSQL, se escoge esta opción.

\subsection{ORM}
\begin{itemize}
    \item TypeORM
    \item Sequelize
    \item Prisma
\end{itemize}

\subsection{Almacenamiento de archivos}
\begin{itemize}
    \item Amazon S3
    \item DigitalOcean Spaces
    \item Google Cloud Storage / Firestore Storage
\end{itemize}

\subsection{IDE}
\begin{itemize}
    \item VSCode
    \item JetBrains
\end{itemize}

\subsection{Gestor de paquetes}
\begin{itemize}
    \item npm
    \item yarn
    \item pnpm
\end{itemize}

\subsection{Documentación}
\begin{itemize}
    \item Docusaurus
    \item GitBook
    \item VuePress
\end{itemize}


% Se explican brevemente, con logos y capturas ilustrativas si es necesario, 
% y se proporcionan tablas comparativas. No se decide todavía cuáles elegir, 
% luego en la propuesta
