\chapter{Implementación}

\section{Metodología de trabajo}
% Ciclo de vida del software, prototipado evolutivo, metodologías
% ágiles, revisión bibliográfica, experimento o estudio de casos,
% prueba de concepto, etc.

Se va a usar una metodología ágil usando el \textit{framework} Scrum adaptado a un equipo unipersonal.


\subsection{Planificación temporal}

El periodo desde la asignación del trabajo (29 de octubre de 2021) hasta el inicio del primer sprint se dedicará a revisar documentación sobre posibles tecnologías a usar, probándolas siempre que sea posible. Antes del inicio del primer sprint se tendrá:

\begin{itemize}
    \item Una base del código del proyecto montada y lista para desplegar en un servidor.
    \item Una decisión de las tecnologías a usar.
    \item Una decisión sobre las herramientas a usar durante el desarrollo (gestión de tareas, gestión de versiones para el código, gestión de versiones para la documentación).
\end{itemize}

A partir del 21 de marzo de 2022, el trabajo simultáneo de desarrollo y escritura de la memoria se dividirá en cuatro sprints:
\begin{itemize}
    \item Primer sprint: 21 de marzo a 1 de abril
    \item Segundo sprint: 18 de julio a 29 de julio
    \item Tercer sprint: 3 de octubre a 12 de octubre
    \item Cuarto sprint: 17 de octubre a 28 de octubre
\end{itemize}

El periodo entre el 28 de octubre y la entrega del trabajo se dedicará a la revisión de la memoria y el análisis de uso de la aplicación en un caso de uso real.


\section{Presupuesto}
% Cuánto costaría llevar a cabo este proyecto en la realidad

% Incluir: costes de personal, de equipamiento (hardware y
% software), otros (desplazamientos, difusión, patentes o
% registros de propiedad, etc.)
% Costes de personal: por horas o meses. Incluir gastos de
% seguros sociales y cuota patronal. Buscar y citar una
% referencia fiable en la que basaros.


\section{Herramientas y tecnologías escogidas}


\subsection{Lenguaje de programación}
Typescript

\subsection{Framework}
GrammY

\subsection{Base de datos}
PostgreSQL

\subsection{ORM}
Prisma

\subsection{IDE}
JetBrains

\subsection{Almacenamiento de archivos}
Amazon S3






% Más subsecciones dependiendo de la metodología empleada


% Debe ser la sección más larga del documento.
% Usar herramientas de Ingeniería: diagramas, esquemas, plantillas, etc.

% - Incluir un diagrama de la arquitectura del sistema (visión general).
% - Indicar qué herramientas de la sección anterior se eligen para el
% desarrollo, por qué y para qué.
% - Poner solo pequeños trozos de código si es necesario para explicar
% detalles de implementación o justificar decisiones del diseño o de la
% tecnología.


\section{Primer sprint}

Analizador de comandos
Unirse a agrupación,
Lógica de negocio para los modelos
Middleware para plantillas


\section{Segundo sprint}

Middleware para información de usuario/cuenta
Crear evento, eliminar evento
Crear agrupación, eliminar agrupación
Crear membresía, eliminar membresía
Habilitar/deshabilitar enlace de invitación
Crear evento, eliminar evento

Contribuciones a GrammY: Prisma Adapter, conversations issue, calendario
