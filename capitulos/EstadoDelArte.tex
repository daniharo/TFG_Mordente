\chapter{Estado del arte}

% Descripción de los antecedentes del trabajo.
% Elegir un título apropiado.

\section{Dominio del problema a resolver}





Para la correcta planificación de ensayos y eventos, tanto el director como los administradores necesitan conocer con antelación qué musicos tienen pensado asistir y cuáles estarán ausentes. Esto les permitirá saber si por ejemplo se necesita llamar a un músico externo a la agrupación para que participe en un evento.

% Descripción más detallada del problema, carencias que existen,
% características específicas de los usuarios finales, descripción de
% trabajos previos del cual este es una extensión o de proyectos de 
% investigación en el que se enmarca, etc.



\section{Metodologías y tecnologías de base que podrían usarse}

\subsection{Lenguaje de programación}

\begin{itemize}
    \item JavaScript
    \item TypeScript
    \item Python
\end{itemize}

\subsection{Framework para el desarrollo del bot}
\begin{itemize}
    \item Telegraf
    \item node-telegram-bot-api
    \item python-telegram-bot
\end{itemize}

\subsection{Base de datos}
\begin{itemize}
    \item Relacional
    \begin{itemize}
        \item PostgreSQL
        \item MariaDB
        \item MySQL
    \end{itemize}
    \item No relacional
    \begin{itemize}
        \item MongoDB
        \item Firestore
    \end{itemize}
\end{itemize}

\subsection{ORM}
\begin{itemize}
    \item TypeORM
    \item Sequelize
    \item Prisma
\end{itemize}

\subsection{Almacenamiento de archivos}
\begin{itemize}
    \item Amazon S3
    \item DigitalOcean Spaces
    \item Google Cloud Storage / Firestore Storage
\end{itemize}

\subsection{IDE}
\begin{itemize}
    \item VSCode
    \item JetBrains
\end{itemize}

\subsection{Gestor de paquetes}
\begin{itemize}
    \item npm
    \item yarn
    \item pnpm
\end{itemize}

\subsection{Documentación}
\begin{itemize}
    \item Docusaurus
    \item GitBook
    \item VuePress
\end{itemize}




% Se explican brevemente, con logos y capturas ilustrativas si es necesario, 
% y se proporcionan tablas comparativas. No se decide todavía cuáles elegir, 
% luego en la propuesta

\section{Trabajos relacionados}
% Evidencias científicas o revisiones recientes de uso de tecnologías o
% metodologías para resolver un problema igual o similar al mío. Buscar
% referencias en fuentes fiables como Google Scholar.

\section{Aplicaciones similares}
% Descripciones, capturas y tabla comparativa
Alternativas: Glissandoo, Whatsapp

En los últimos años ha surgido software dedicado a este propósito, como es el caso de \textit{Glissandoo}. Sin embargo, esta herramienta tiene varias desventajas:

\begin{itemize}
    \item El código es privado, y una agrupación no puede montar su propio servidor con la aplicación en cuestión.
    \item Es de pago para uso en agrupaciones con más de 20 miembros.
    \item Solo se puede usar en móvil, ya que no dispone de versión web ni de escritorio. (Esto ha cambiado durante la realización de este trabajo, ya que han implementado una versión web)
    \item Sigue existiendo duplicidad, ya que aunque los miembros usen esta aplicación, la comunicación bidireccional entre administradores y miembros de la banda se sigue realizando a través de una herramienta distinta de comunicación no especializada.
\end{itemize}

En este trabajo se propone una alternativa libre que pueda resolver los problemas anteriores, de modo que:

\begin{itemize}
    \item Cada agrupación podrá montar su propio servidor con la aplicación, personálizandola y adaptándola a sus necesidades.
    \item Los gastos por uso de la herramienta solo dependerán del servidor que use la agrupación, si usa un servidor dedicado.
    \item Será usable en todas las plataformas en las que se pueda acceder a un navegador web.
    \item Estará integrada en una herramienta de comunicación no especializada como es Telegram, pero añadiendo funcionalidad específica y automática necesaria para la gestión de la agrupación, eximiendo a los administradores y el director de tareas que se pueden automatizar.
\end{itemize}

