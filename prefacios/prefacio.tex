\chapter*{}
%\thispagestyle{empty}
%\cleardoublepage

%\thispagestyle{empty}

\input{portada/portada_2}



\cleardoublepage
\thispagestyle{empty}

\begin{center}
{\large\bfseries \myTitle}\\
\end{center}
\begin{center}
\myName\\
\end{center}

%\vspace{0.7cm}
\noindent{\textbf{Palabras clave}: Telegram, bot, gestión, música, agrupación, banda}\\

\vspace{0.7cm}
\noindent{\textbf{Resumen}}\\

% De media página a una página
% Incluir:
% – motivación o problema a resolver
% – características relevantes de la solución que se aporta

Dirigir o administrar una formación musical es una tarea que crece en complejidad en cuanto aumentan los miembros de la agrupación. Es necesaria una solución de código libre y gratuita que centralice todas estas tareas y facilite el trabajo de los responsables de una agrupación musical. A través de la herramienta, en este caso un bot de Telegram, las sociedades musicales dispondrán de un innovador sistema para el control y la previsión de asistencia, la distribución interna de repertorio y una potente herramienta de utilidad para los miembros.

Así pues, el objetivo de este proyecto es crear un bot que sea fácil de instanciar en distintas agrupaciones. Ofrecerá utilidades a sus miembros como recordatorios de ensayos, notificaciones de eventos, gestión de asistencia y visualización de las obras.

Como consecuencia de este trabajo también descubriremos si una interfaz de usuario basada solamente en un bot de Telegram ofrece suficientes posibilidades para que los usuarios la perciban como intuitiva y usable sin acompañarse de una aplicación web o multiplataforma.


\cleardoublepage


\thispagestyle{empty}


\begin{center}
{\large\bfseries Telegram Bot for Managing Musical Ensembles}\\
\end{center}
\begin{center}
Daniel Haro Contreras\\
\end{center}

%\vspace{0.7cm}
\noindent{\textbf{Keywords}: Telegram, bot, management, music, ensemble, band}\\

\vspace{0.7cm}
\noindent{\textbf{Abstract}}\\

The task of managing a musical ensemble grows in complexity as the number of members increases. An open-source and free solution is needed to centralize all these tasks and facilitate the work of those responsible for a musical group. Through the tool, in this case a Telegram bot, musical ensembles will have an innovative system for controlling and forecasting attendance, the internal distribution of repertoire and a powerful tool for members.

Thus, the goal of this project is to create a bot that is easy to instantiate in different groups. It will offer utilities to its members such as rehearsal reminders, event notifications, attendance management and visualization of sheet music.

As a consequence of this work we will also find out whether a user interface based only on a Telegram bot offers enough possibilities for users to perceive it as intuitive and usable without being accompanied by a web or cross-platform application.

\chapter*{}
\thispagestyle{empty}

\noindent\rule[-1ex]{\textwidth}{2pt}\\[4.5ex]

Yo, \textbf{Daniel Haro Contreras}, alumno de la titulación TITULACIÓN de la \textbf{Escuela Técnica Superior
de Ingenierías Informática y de Telecomunicación de la Universidad de Granada}, con DNI 76656133P, autorizo la
ubicación de la siguiente copia de mi Trabajo Fin de Grado en la biblioteca del centro para que pueda ser
consultada por las personas que lo deseen.

\vspace{6cm}

\noindent Fdo: Daniel Haro Contreras

\vspace{2cm}

\begin{flushright}
Granada a 16 de noviembre de 2022.
\end{flushright}


\chapter*{}
\thispagestyle{empty}

\noindent\rule[-1ex]{\textwidth}{2pt}\\[4.5ex]

D.\textsuperscript{a} \textbf{\myProf}, profesora del Departamento de Lenguajes y Sistemas Informáticos de la Universidad de Granada.


\vspace{0.5cm}

\textbf{Informa:}

\vspace{0.5cm}

Que el presente trabajo, titulado \textit{\textbf{\myTitle}},
ha sido realizado bajo su supervisión por \textbf{\myName}, y autorizo la defensa de dicho trabajo ante el tribunal
que corresponda.

\vspace{0.5cm}

Y para que conste, expiden y firman el presente informe en Granada a 16 de noviembre de 2022.

\vspace{1cm}

\textbf{La directora:}

\vspace{5cm}

\noindent \textbf{\myProf}

\chapter*{Agradecimientos}
\thispagestyle{empty}

       \vspace{1cm}


\textit{Este camino de aprendizaje comenzó hace cuatro años llenos de dificultades, altibajos, miedos, ilusiones y esfuerzo. Mi gratitud a los profesores y profesoras que me han guiado en el camino por hacer que todo ello merezca la pena.}

\textit{Agradezco a mis compañeros y amigos de la Banda de la Agrupación Músico-Cultural ``San Sebastián'' de Padul su total disposición para colaborar en las pruebas de este proyecto.}

\textit{Por último, expreso mi agradecimiento a mi familia y pareja, que me han acompañado en esta dura y fructífera travesía haciéndolo todo más llevadero.}
